\documentclass{article}

% Language setting
% Replace `english' with e.g. `spanish' to change the document language
\usepackage[english]{babel}

% Set page size and margins
% Replace `letterpaper' with `a4paper' for UK/EU standard size
\usepackage[letterpaper,top=2cm,bottom=2cm,left=3cm,right=3cm,marginparwidth=1.75cm]{geometry}

% Useful packages
\usepackage{amsmath}
\usepackage{graphicx}
\usepackage[colorlinks=true, allcolors=blue]{hyperref}

\title{Your Paper}
\author{You}

\begin{document}
\maketitle

\begin{abstract}
Your abstract.
\end{abstract}

\section{Methods}
\subsection{Dimensionality Reduction}
I used three methods for the reduction of dimensionality.
\begin{enumerate}
    \item PCA
    \item t-SNE
    \item UMAP
\end{enumerate}

\subsection{Clustering}
I used three methods for clustering.
\begin{enumerate}
    \item K-Means
    \item DBSCAN
    \item Gaussian Mixture
\end{enumerate}

For methods that are given the expected number of clusters, I have iterated over a range of possible numbers, calculating the silhouette score on each iteration, and updating a stored value of the maximum score. While the minimum number of clusters is naturally 2, the maximum number is a matter of guessing TODO.

For the DBSCAN algorithm, which is not based on calculating centroids, I have used the same parameter for calculating the epsilon parameter, by a heuristic function which divides the total area of the set of data points by this increasing factor, yielding a decreasing set of distances, which are used as the epsilon input parameter.  

\url{https://www.overleaf.com/contact}.

\bibliographystyle{alpha}
\bibliography{sample}

\end{document}